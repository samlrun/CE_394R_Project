\documentclass[3p, authoryear]{elsarticle} %review=doublespace preprint=single 5p=2 column
%%% Begin My package additions %%%%%%%%%%%%%%%%%%%
\usepackage[hyphens]{url}

  \journal{Findings} % Sets Journal name


\usepackage{lineno} % add
\providecommand{\tightlist}{%
  \setlength{\itemsep}{0pt}\setlength{\parskip}{0pt}}

\usepackage{graphicx}
%%%%%%%%%%%%%%%% end my additions to header

\usepackage[T1]{fontenc}
\usepackage{lmodern}
\usepackage{amssymb,amsmath}
\usepackage{ifxetex,ifluatex}
\usepackage{fixltx2e} % provides \textsubscript
% use upquote if available, for straight quotes in verbatim environments
\IfFileExists{upquote.sty}{\usepackage{upquote}}{}
\ifnum 0\ifxetex 1\fi\ifluatex 1\fi=0 % if pdftex
  \usepackage[utf8]{inputenc}
\else % if luatex or xelatex
  \usepackage{fontspec}
  \ifxetex
    \usepackage{xltxtra,xunicode}
  \fi
  \defaultfontfeatures{Mapping=tex-text,Scale=MatchLowercase}
  \newcommand{\euro}{€}
\fi
% use microtype if available
\IfFileExists{microtype.sty}{\usepackage{microtype}}{}
\usepackage{natbib}
\bibliographystyle{apalike}
\usepackage{longtable,booktabs,array}
\usepackage{calc} % for calculating minipage widths
% Correct order of tables after \paragraph or \subparagraph
\usepackage{etoolbox}
\makeatletter
\patchcmd\longtable{\par}{\if@noskipsec\mbox{}\fi\par}{}{}
\makeatother
% Allow footnotes in longtable head/foot
\IfFileExists{footnotehyper.sty}{\usepackage{footnotehyper}}{\usepackage{footnote}}
\makesavenoteenv{longtable}
\ifxetex
  \usepackage[setpagesize=false, % page size defined by xetex
              unicode=false, % unicode breaks when used with xetex
              xetex]{hyperref}
\else
  \usepackage[unicode=true]{hyperref}
\fi
\hypersetup{breaklinks=true,
            bookmarks=true,
            pdfauthor={},
            pdftitle={Vehicle Crash Severity Correlating Factors},
            colorlinks=false,
            urlcolor=blue,
            linkcolor=magenta,
            pdfborder={0 0 0}}
\urlstyle{same}  % don't use monospace font for urls

\setcounter{secnumdepth}{5}
% Pandoc toggle for numbering sections (defaults to be off)

% Pandoc citation processing

% Pandoc header
\usepackage{booktabs}
\usepackage{booktabs}
\usepackage{longtable}
\usepackage{array}
\usepackage{multirow}
\usepackage{wrapfig}
\usepackage{float}
\usepackage{colortbl}
\usepackage{pdflscape}
\usepackage{tabu}
\usepackage{threeparttable}
\usepackage{threeparttablex}
\usepackage[normalem]{ulem}
\usepackage{makecell}
\usepackage{xcolor}



\begin{document}
\begin{frontmatter}

  \title{Vehicle Crash Severity Correlating Factors}
    \author[Brigham Young University]{Samuel Runyan\corref{1}}
   \ead{samyan@byu.edu} 
      \address[Brigham Young University]{Civil and Environmental Engineering Department, 430 Engineering Building, Provo, Utah 84602}
    
  \begin{abstract}
  This is where the abstract should go.
  \end{abstract}
   \begin{keyword} Highway Safety Analysis Crash Severity Statistical Correlation\end{keyword}
 \end{frontmatter}

\hypertarget{intro}{%
\section{QUESTIONS}\label{intro}}

Research in network screening for highway safety analysis is continually developing and improving as data processing technologies and understanding of crash data improve. However, a major difficulty with most current network screening methodologies is that they do not account for the influence of crash severity towards the overall potential for safety improvement (PSI) of roadways. Therefore, it is valuable to use a crash severity-weighted approach when performing network screening. \cite{yasmin2018} and \cite{afghari2020} assert that a joint crash count with crash severity model is best suited to identify sites with the most PSI because of the correlation between crash counts and crash severity. In addition to this joint model, it may be helpful to consider other factors which may contribute to the crash severity distribution at a given site. \cite{ghadi2020} attempted to determine the impact of spatial and environmental factors on crash severity, but few others have added to this research.

The purpose of this article is to identify any additional factors that may correlate to crash severity such as the manner of collision, the presence of pedestrians, the light conditions, and so on. The results of this article will help determine which of these factors are significant enough to include in severity-weighted network screening. In the past, many network screening models have been limited by precendent, but this research aims to expound on and improve these tried and tested methods. With better understanding of how crash factors relate to crash severity, it may be possible to more proactively prevent severe crashes rather than waiting for them to happen before they can be modeled.

\hypertarget{methods}{%
\section{METHODS}\label{methods}}

In this chapter, you describe the approach you have taken on the problem. This
usually involves a discussion about both the data you used and the models you
applied.

\hypertarget{data}{%
\subsection{Data}\label{data}}

Discuss where you got your data, how you cleaned it, any assumptions you made.

Often there will be a table describing summary statistics of your dataset.
Table \ref{tab:datasummary} shows a nice table using the \href{https://vincentarelbundock.github.io/modelsummary/articles/datasummary.html}{\texttt{datasummary}}
functions in the \texttt{modelsummary} package.

\begin{table}

\caption{\label{tab:datasummary}Descriptive Statistics of Dataset}
\centering
\begin{tabular}[t]{llrrrrrrrrrrrr}
\toprule
\multicolumn{2}{c}{ } & \multicolumn{2}{c}{regcar (N=10930)} & \multicolumn{2}{c}{sportuv (N=1048)} & \multicolumn{2}{c}{sportcar (N=880)} & \multicolumn{2}{c}{stwagon (N=4446)} & \multicolumn{2}{c}{truck (N=5628)} & \multicolumn{2}{c}{van (N=4992)} \\
\cmidrule(l{3pt}r{3pt}){3-4} \cmidrule(l{3pt}r{3pt}){5-6} \cmidrule(l{3pt}r{3pt}){7-8} \cmidrule(l{3pt}r{3pt}){9-10} \cmidrule(l{3pt}r{3pt}){11-12} \cmidrule(l{3pt}r{3pt}){13-14}
  &    & Mean & Std. Dev. & Mean  & Std. Dev.  & Mean   & Std. Dev.   & Mean    & Std. Dev.    & Mean     & Std. Dev.     & Mean      & Std. Dev.     \\
\midrule
price &  & 4.2 & 1.9 & 4.7 & 1.9 & 4.8 & 2.2 & 4.1 & 1.9 & 4.2 & 2.0 & 4.2 & 1.9\\
range &  & 237.2 & 94.5 & 241.6 & 94.7 & 233.6 & 96.7 & 238.7 & 94.3 & 238.2 & 93.1 & 236.8 & 94.7\\
size &  & 2.4 & 0.8 & 2.1 & 1.0 & 1.4 & 1.0 & 2.3 & 0.8 & 2.4 & 0.8 & 2.5 & 0.7\\
\midrule
 &  & N & Pct. & N & Pct. & N & Pct. & N & Pct. & N & Pct. & N & Pct.\\
fuel & gasoline & 2704 & 24.7 & 280 & 26.7 & 218 & 24.8 & 1096 & 24.7 & 1413 & 25.1 & 1247 & 25.0\\
 & methanol & 2729 & 25.0 & 246 & 23.5 & 225 & 25.6 & 1091 & 24.5 & 1445 & 25.7 & 1216 & 24.4\\
 & cng & 2767 & 25.3 & 260 & 24.8 & 238 & 27.0 & 1109 & 24.9 & 1360 & 24.2 & 1282 & 25.7\\
 & electric & 2730 & 25.0 & 262 & 25.0 & 199 & 22.6 & 1150 & 25.9 & 1410 & 25.1 & 1247 & 25.0\\
\bottomrule
\end{tabular}
\end{table}

\hypertarget{models}{%
\subsection{Models}\label{models}}

If your work is mostly a new model, you probably will have introduced some
details in the literature review. But this is where you describe the
mathematical construction of your model, the variables it uses, and other
things. Some methods are so common (linear regression) that it is unnecessary to
explore them in detail. But others will need to be described, often with
mathematics. For example, the probability of a multinomial logit model is

\begin{equation}
  P_i(X_{in}) = \frac{e^{X_{in}\beta_i}}{\sum_{j \in J}e^{X_{jn}\beta_j}}
  \label{eq:mnl}
\end{equation}

Use \href{https://www.overleaf.com/learn/latex/mathematical_expressions}{LaTeX mathematics}.
You'll want to number display equations so that you can
refer to them later in the manuscript. Other simpler math can be described inline,
like saying that \(i, j \in J\). Details on using equations in bookdown are available
\href{https://bookdown.org/yihui/bookdown/markdown-extensions-by-bookdown.html}{here}.

\hypertarget{findings}{%
\section{FINDINGS}\label{findings}}

This section might be called ``Results'' instead of ``Applications,'' depending
on what it is that you are working on. But you'll probably say something like
``The initial model estimation results are given in Table \ref{tab:estimation-results}.''
That table is created with the \texttt{modelsummary()} package and function.

With those results presented, you can go into a discussion of what they mean.
first, discuss the actual results that are shown in the table, and then any
interesting or unintuitive observations.

\hypertarget{additional-analysis}{%
\subsection{Additional Analysis}\label{additional-analysis}}

Usually, it is good to use your model for something.

\begin{itemize}
\tightlist
\item
  Hypothetical policy analysis
\item
  Statistical validation effort
\item
  Equity or impact analysis
\end{itemize}

If the analysis is substantial, it might become its own top-level section.

\bibliography{book.bib}


\end{document}
